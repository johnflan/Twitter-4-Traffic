\subsection{Clustering}

\subsection{Gamification}

\subsection{Analysis of other sources}
As the aim of this project is providing information to the user about traffic disruptions, more sources can 
easily be used for acquiring more information about disruptions in London and other areas. Such examples 
of sources would be news reports published in various websites. These reports could include accidents, general traffic or even 
bad road conditions due to the weather. Moreover, there exist TfL Twitter bots that frequently tweet
about traffic disruptions. Another source that can be used is Facebook. Data regarding traffic can be
extracted from posts, comments or even public groups whose current location is London.

\subsection{Enhanced tweet geocoding}
For some reasons, mobile users cannot tweet the right position where the disruption is. Geolocation needs be double-checked by comparing with geography data stored in server before inserted into database. What's more, some geolocations of street stored in the database will change in the future. Those data should be checked and updated periodically. Google maps geocoding function can be implemented to provide the latest information for those street addresses. Algorithm for extracting street address from text need to be optimized to get more accurate results. Then those results can be used to receive responding longitude and latitude from Google Map. Meanwhile, database can be expanded to receive more tweets from Twitter for application users to provide traffic disruptions even out of London.