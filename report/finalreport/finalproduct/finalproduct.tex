\subsection{Achieved goals and difficulties}
The specifications and which of them were implemented or not is described in
Section 2. All the features from the minimum specifications were imlemented
fully. Also, there were features that were not in the specifications and were
implemented nonetheless.

A profanity checker was included, as stated in Section 4.2.1, to check if the social data included any
cursing words. Every time a tweet is processed before entering the database
there is a profanity check and it is marked accordingly. By default the
application does not use the profanity checker, but the user can enable it from
the settings.

Soundex algorithm was implemented as well, as stated in Section 4.2.2, in order to enhance the possibility to identify
a location just from the text of a tweet. This algorithm understands
misspelled words, and so gives the ability to search for street names that may
be mistyped.

The reasons some features were not included in the final project is
that there were some difficulties that could not be overcome, or the priorities
changed.
The reasons they were not implemented are described below:

\emph{Identify traffic disruptions from Twitter}

The reason that this feature could not be included is that the traffic tweets
for a small timeslice are too few to get results from the clustering described
in Section 7.1. The feature is still implemented, but the data used to test
were in a timeslice of an hour and using data that were not specifically about
traffic. This means that a new cluster we may find might also indicate an
event that is irrelevant to traffic. Nonetheless, this should prove very useful
in the future.

\emph{Enhance clustering algorithm}

This feature could not be included in the product of this project because we
could not implement clustering to work in real time, as described in the
previous feature that was not implemented.

\emph{Present tweets on the map}

This feature was not implemented, because the map would get too crowded if all
the tweets about traffic were visible. It would only help the user if there
were clustered tweets and there was one event generated from the specific
cluster, but as stated above clustering was not implemented.
