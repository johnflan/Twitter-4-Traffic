\subsubsection{Data acquisition}
The sources that we collect data from are Transport for London (TfL) and Twitter.

The TfL website provides different types of data feeds that can be used to obtain information relevant to traffic. 
One of the TfL data feeds we use is the "live traffic disruptions". This provides information about traffic disruption events in the area of London. This information contains details about the severity of the event, the type (e.g. road works, signal failure, accident), the location and an estimated time for the event's end.

To provide a more complete view around the area of the event we also store the "live traffic camera images" feed from TfL. This provides urls to pictures taken by traffic cameras within the last 15 minutes as well as the location of the camera and the timestamp when the picture was taken.

These feeds are returned in XML format and need to be parsed before we store them in the database. To achieve this, we use the xml.dom.minidom python module. This can parse the XML file to a Document Object Model (DOM). A DOM tree can be accessed by using functions to call each field of the XML object that was used to create it.

For Twitter we defined the area around London that we are interested to collect tweets from. In addition to this, we used a search query with the words that we want the results from Twitter to contain. This query is:  «traffic OR accident OR tailback OR gridlock OR m25 OR standstill OR road OR street OR stuck OR car OR bus OR delay». 

In contrast to the TfL, the Twitter data are returned in a JSON format. For this reason, we used the python-twitter API which saves the JSON information into classes and variables that can then be easily accessed and stored in the database.

\subsubsection{Data analysis}
A big and crucial part of the project was the data analysis. Specifically for the needs of the project a sub category of the data analysis, which is called text classification, has been implemented. Text classification is a way to categorize documents or pieces of text. By examining the word usage in a piece of text, classifiers can decide what class label to assign to it. A binary classifier decides between two labels, such as traffic or non-traffic. The text can either be one label or the other, but not both, whereas a multi-label classifier can assign one or more labels to a piece of text.

Classification works by learning from labelled feature sets, or training data, to later classify an unlabelled feature set. A feature set is basically a key-value mapping of feature names to feature values. In the case of text classification, the feature names are usually words, and the values are all True. As the documents may have unknown words, and the number of possible words may be very large, words that don't occur in the text are omitted, instead of including them in a feature set with the value False.

For the project it was essential to classify the tweets that the Search API was fetching, as traffic or as non-traffic. The non-traffic tweets are being rejected while the tweets about traffic are being stored in the database, in order to be correlated with the current disruption and then presented to the application. For this purpose several classification techniques have been implemented and tested. To be more specific, the team has investigated the classification of the tweets with the methods Support Vector Machines and Naive Bayes. 

\textbf{Naive Bayes}

Given a set of objects, each of which belongs to a known class, and each of which has a known vector of variables, the aim is to construct a rule which will allow the assigning of future objects to a class, given only the vectors of variables describing the future objects. Problems of this kind, called problems of supervised classification, are ubiquitous, and many methods for constructing such rules have been developed. One very important method is the Naive Bayes Reasoning. This is a well- established Bayesian method primarily formulated for performing classification tasks. Given its simplicity, i.e., the assumption that the independent variables are statistically independent, Naive Bayes models are effective classification tools that are easy to use and interpret. Naive Bayes is particularly appropriate when the dimensionality of the independent space. For the reasons given above, Naive Bayes can often outperform other more sophisticated classification methods. A variety of methods exist for modelling the conditional distributions of the inputs including normal, lognormal, gamma, and Poisson. 

This classifier has been created using the Bag of Words model and the NLTK suites of libraries. NLTK is the Natural Language Toolkit, a comprehensive Python library for natural language processing and text analytics. The group decided to use the Natural Language Toolkit because of its simplicity, consistency, extensibility and its modularity. Additionally some of the members had already experience with it and they were aware of its efficient and its good classification results. Furthermore, it was decided the usage of the Bag of Words feature extraction. Text feature extraction is the process of transforming what is essentially a list of words into a feature set that is usable by a classifier. The NLTK Naïve Bayes classifier expects dictionary style feature sets, so the text should be transformed into a dictionary. The Bag of Words model is a well-known method for representing documents, which ignores the word orders. It constructs a word dictionary from all the words of an instance where every word gets the value True. An instance is a single feature set. It represents a single occurrence of a combination of features. A labelled feature set is in fact an instance with a known class label that we can use for training or evaluation.

As it has been mentioned before, for the training of the classifier it’s required a labelled data. To accomplish that, a simple script was created. This script was being executed on a temporary table on the database which was containing raw, unlabelled, tweets. During the execution it was presenting random tweets from this table and the user was able to press four buttons in order to label the tweets as traffic (personalized tweets about traffic), non-traffic, unclear and bot (tweets about traffic from official sites). After the gathering of a big amount of traffic-about tweets, the table in which the labelled tweets were being stored was used to train the classifier. However, before this data was used to train the classifier, a several linguistic normalization has been applied on the labelled text.  More details for the normalization will be described in a next section.

\textbf{Support Vector Machines}

The second supervised learning method that it was fully integrated and tested is the Support Vector Machines (SVM). This is a method that performs regression and classification tasks by constructing nonlinear decision boundaries. Because of the nature of the feature space in which these boundaries are found, Support Vector Machines can exhibit a large degree of flexibility in handling classification and regression tasks of varied complexities. There are several types of Support Vector models including linear, polynomial, RBF, and sigmoid.

\textbf{Normalization}

For both of the above methods a separate class has been created to be used for the tweets normalization. Normalization is the way to eliminate the low information features. Eliminating low information features gives to the model clarity by removing noisy data. Additionally it reduces the possibility to get over-fitting. By using the higher information features, the performance is increasing while the size of the model is decreasing, which results in less memory usage along with faster training and classification. This normalization was being applied on the labelled tweets before the classifier training and is continue being applied on the newly fetched tweets as well. Using this pre-processing on the tweets the accuracy of the classifier has been increased. On the following paragraphs we will present the normalization techniques which have been adopted. 

The first step was to convert the current links to a more readable way. This has been done by using a regular expression to recognize the link. After that the domain is being extracted from the url and it replaces the link itself. The next step is to replace the emoticons with a global name so they will not be deleted during the punctuation removal. That’s important because the emoticons offer useful information during the classification. For this purpose, the team has integrated a script which is responsible to find those emoticons, assign them to a group of emoticons and replace them with the name of the group. Four groups of emoticons have been created: Happy, Sad, Very Happy, Very Sad. The next step is to remove the punctuations. The team has accomplished that by creating a regular expression which represents all the possible punctuations. 

After the above linguistic transformations, the tweets are being tokenized into words. Then the unnecessary special words, like the usernames, are being removed from the tweets. Subsequently, lemmatization has been applied on the tokenized data by removing and replacing word suffixes to arrive at a common root form of the word in order to group up the common words. This method has been chosen over the stemming because lemma is a canonical a set of words, instead of the stem which in many cases is not a real world. Afterwards several stopwrods are being removed from the tweet because they are so common that are practical meaningless. However because of the natural of the tweets of being a natural and unstructured language, only a small amount of words has been declared as stopwrods. The last but not least step of the text normalization is the tracking of the bigrams collocations from the tweets. Because the bigrams less common than most individual words, including them in the Bag of Words increases the classifier accuracy.

\subsubsection{Storage}
For the storage of data we use a PostgreSQL database. Because of the high amount of database queries we needed to use an efficient way to store and analyze the information we acquired from the tfl feeds and twitter. For this reason, we decided to use a Geographic Information System called PostGIS. This allowed us to store geolocations (longitude and latitude) as points.

With the use of PostGIS queries to the database became easier and more efficient. The use of functions provided by PostGIS, such as ST\_DWithin to find all points around a route or a point within a radius, or ST\_Distance to find the distance between two points, made all queries simpler. In addition, the main advantage of PostGIS is that it uses generalized search trees (GiST) to index the geometries. 

A GiST like a B-tree uses key-pointer pairs. The difference with B-trees is that a GiST key is a user-defined data type. This allows different types of operations, rather than simple comparisons, such as nearest-neighbor searches and statistical approximations over large data sets. In PostGIS, the GiST is used for spatial indexing by allowing the index to use a bounding box for each geometry (e.g. line) instead of storing the whole geometry in it. Therefore, with bounding box comparison, instead of comparing geometries, functions such as ST\_DWithin are made more efficient.

For the project we used the following tables:
\begin{itemize}
\item Tables used by the final application
  \begin{itemize}
  \item tfl: The current tfl disruption events
  \item archive: To store old tfl disruption events for further analysis
  \item tflcameras: The current camera pictures' url
  \item tweets: To store the traffic tweets acquired from twitter
  \item geolookup: Used to store addresses with their SoundEX value and their geolocation which is acquired from Google Maps
  \item tweets\_metrics: Used to store different types of metrics for the tweets
  \end{itemize}
\item Tables used to train the classifier
  \begin{itemize}
  \item labelled\_tweets: To store manually labelled tweets as traffic or non-traffic
  \item stop\_words: Words that can be removed before data analysis
  \end{itemize}
\item Tables used by PostGIS
  \begin{itemize}
  \item geography\_columns: This is a view that shows all the columns of the database that use geography points
  \item spatial\_ref\_sys: A list of spatial reference systems and details to transform between them
  \end{itemize}
\end{itemize}

\subsubsection{Interface}

Before the actual server was implemented, the creation of a mock server api was essential. This 
provided us with the ability to seperate the work on the application and server design from the very 
beginning of the project. At this stage, the rest endpoints had to be defined in order to agree on the 
data format that was transferred between the server and the client. The next thing was creating mocked 
JSON data for responding to the requests. Such data included fixed sets of disruptions, tweets and 
cameras. 





