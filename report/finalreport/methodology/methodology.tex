\subsection{Development approach} In the start of the project the team was disgussing about which 
development methodologies should be applied. It was decided amongst the group that combining Scrum 
and XP characteristics would be the most effective way merely because Scrum focuses more on the 
project management side whereas XP more on the programming part.\\ 
Firstly, the basic requirements were agreed with the team supervisor but also some inspirational 
features that were considered to implemented later on. Managing the development lifecycle of the 
project was very important for the team. Meetings with the supervisor took place each week where the 
project progress was thoroughly discussed. In addition, various issues were mentioned in order to 
acquire solutions. Except from the meetings with the supervisor, the team had its own meeting every 
week for keeping up with what every one has been doing for the last weeek and also new tasks were 
assigned to each member. An agenda for each meating was stored in an online document and tasks were 
assigned and managed using an online visual board to encourage a more agile development.\\ 
As regards to the XP approach, it was agreed that adopting some of its techniques would 
significantly improve the development process. More specifically, pair programming was effectively 
put in use. Team members where working in pairs whenever possible to develop a single feature. 
Additionally, XP promotes test-driven development. As mentioned in the first report, testing would 
be rather difficult due to the nature of the project being partly research based. However, essential 
tests were implemented in the server to validate the classifier as well as blackbox tests for the 
responsiveness of the rest api server. \\
Throughout the development, various technologies had to be 
used and implemented. According to each members skills these tasks were divided in such ways to 
effectively use the members previous experience. The team was also split to different design aspects 
of the project, however more on the team structure will be discussed further on.\\ 
Lastly, for achieving parallel development between the mobile client application and the server, a 
mock server API was created, as mentioned in earlier reports, that was accepting http requests and 
was returning static data sets to the client. 

\subsection{Testing} 
\subsubsection{Classifier Evaluation} 
\subsubsection{Functional\\Integration Testing} 
\subsubsection{Unit Testing} 

