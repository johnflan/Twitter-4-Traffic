\subsection{Development approach} 
To begin identifying an appropriate development methodology, it was necessary to firstly understand 
the project requirements, timescale of the project and the team members experience. The project requirements and features were agreed over a number of 
meetings with the primary stakeholders.

The two most preferred development methodologies were 
Scrum and XP(Extreme Programming) which are both Agile methodologies. Scrum is an agile project management technique that focuses more 
on the management of software development projects. The product is completed in a series of one to 
four week iterations, or sprints as they are called. Before each sprint, a planning meeting is held 
to determine which features will be implemented during that sprint. Similarly, XP is an agile 
methodology which is designed for small, co-located teams aiming to get quality and productivity as 
high as possible. It does this through the use of rich, short, informal communication paths with 
emphasis on skill, discipline and understanding at the personal level, minimizing all intermediate 
work products. 

It was decided amongst the group that combining characteristics from both methods 
would be most beneficial, merely because they are complementary. Scrum focuses on the project management 
whereas XP focuses on programming.\cite{ScrumXP}

Managing efficiently the development lifecycle of the project was a very 
important task for the team to work effectively. Meetings with the supervisor took place each week, 
the project progress was discussed as well as other possible features that could be implemented. 
In addition, various issues were constantly being brought up in order to provide solutions. Apart from 
the meetings with the supervisor, the team had its own weekly meeting for keeping up with 
everyone's progress for the previous week. New tasks were also assigned to members of the team. For 
every meeting an agenda was stored in an online document containing information about the team 
members absent, location, action items and topics to be discussed. In the end of the meeting tasks 
were assigned, managed or archived using an online visual board, similar to the Scrum Board, to 
encourage a more agile development.

As regards to the XP approach, it was proved that adopting 
some of its techniques would significantly improve the development process. More specifically, pair 
programming was effectively put in use. Team members were working in pairs whenever possible to 
develop a single feature that seemed to be difficult. As a result, any time the two find a section of code 
that appears hard to understand or overly complex, they are to revise it, constantly simplifying and improving it. 
Additionally, XP promotes test-driven development. As mentioned in the 
first report, testing would be rather difficult due to the nature of the project being partly 
research based. However, essential tests were implemented in the server to validate the classifier 
as well as blackbox tests for the correctness and responsiveness of the rest API server. 

Throughout the development, various technologies had to be used and implemented. According to each 
members skills these elements were divided in such a way to effectively use the members previous 
experience. The team was split to concentrate on different design aspects of the project; however, more on the 
team structure will be discussed further on.

Lastly, for achieving parallel development between the mobile client application and the server, 
a mock server API was created, as mentioned in earlier reports that was accepting http requests and 
was returning static data sets to the client. 

\subsection{Testing} 
\subsubsection{Classifier Evaluation} 
Machine learning algorithms induce classifiers that depend on the training set. So there is a need for evaluation and statistical testing to assess the expected error rate of a classification algorithm. Additionally evaluation is crucial to compare the expected error rates of two classification algorithms to be able to say which one is better. Evaluation can also be used as a guide for future improvements on the model. In order to evaluate the classifier several techniques have been used. The first technique is to generate a test set of tweets which their labels are already known. This test set has to be distinct from the train set which has been used to train the classifier. It is then being labelled by the classifier and the labels that it decides are being compared with their correct labels. The second technique is to calculate the accuracy of the classifier which measures the percentage of inputs in the test set that the classifier correctly labelled. To accomplish this, the build-in function of the package NLTK \emph{nltk.classify.accuracy()} has been used.

Additionally techniques have been implemented in order to get moreaccurate evaluations and avoid possible `overfitting'. There is a chance the classifier will become more accurate in the train set and less accurate in the test set with some parameter changes. This is when an "over-fitting" is occurs to the train set. 

The first of these methods is the K-Fold Cross Validation. The dataset is split each time into K equally sized subsets, training and testing datasets, and then in n-th iteration (n=1..k) the n-th subset(testing set) is used for testing the classifier that has been built on all other remaining subsets. To present the result of this method the Confusion Matrix, which is a visualization tool typically used to present the results attained by a learner, has been created. Each column of the matrix represents the instances in a predicted class, while each row represents the instances in an actual class. Thus, the diagonal entries indicate labels that were correctly predicted, and the off-diagonal entries indicate errors. One benefit of a confusion matrix is that it is easy to see if the system is confusing two classes.

The Precision and Recall Rates can be calculated in order to ensure the results from the previous method. Recall describes the completeness of the classification. It is defined as the portion of the traffic tweets versus retrieved by the classifier the number of existing traffic tweets. Precision defines the actual accuracy of the classification. It is defined as the portion of the traffic tweets exist in the total number of tweets classified. The recall and the precision can be derived from the confusion matrix by applying the following formulas:

\[ Precision\textsubscript{A} = tp\textsubscript{A}/(tp\textsubscript{A}+e\textsubscript{BA}+e\textsubscript{CA}) \]

\[ Recall\textsubscript{A} = tp\textsubscript{A}/(tp\textsubscript{A}+e\textsubscript{AB}+e\textsubscript{AC}) \]

Where the values "tp" and "e" are the elements of the confusion matrix as it can been seen on the figure~\ref{fig:confisionMatixCalc}.

\begin{figure}[h!]
\begin{center}
\begin{tabular}{| l || c | c | c | }
    \hline
        & A & B & C  \\ \hline \hline
        A & tp\textsubscript{A} & e\textsubscript{AB} & e\textsubscript{AC} \\ \hline
        B & e\textsubscript{BA }& tp\textsubscript{B} & e\textsubscript{BC} \\\hline
        C & e\textsubscript{CA} & e\textsubscript{CB} & tp\textsubscript{C} \\\hline
    \end{tabular}
	\caption{A simple confusion matrix}
    \label{fig:confisionMatixCalc}
\end{center}
\end{figure}

While recall and precision rates can be individually used to determine the quality of a classifier, it is often more convenient to have a single measure to do the same assessment. The F\textsubscript{1} measure combines the recall and precision rates in a single equation:

\[ F\textsubscript{1} = 2*\frac{precision*recall}{precision+recall} \]

The labelled data was consisting of an unbalance data, 1500 traffic tweets and 14505 non-traffic tweets, two different trainings have been executed and tested. 

Firstly, the classifier was trained with 1000 traffic and 1000 non-traffic tweets. Then it was tested with 500 traffic and 500 non-traffic tweets. The metrics for this training are the following. 

Accuracy of the classifier:   0.870000

Traffic precision:\hspace{15.5 mm}              0.842592592593\\
Traffic recall:\hspace{21.2 mm}                             0.91\\
Traffic F-measure:\hspace{12.8 mm}             0.875\\

Non-Traffic precision:\hspace{7.2 mm}        0.902173913043\\
Non-Traffic recall:\hspace{13 mm}            0.83\\
Non-Traffic F-measure:\hspace{4.6 mm}       0.864583333333\\

\begin{figure}[h!]
\begin{center}
    \begin{tabular}{| l || c | c | }
    \hline
          & Non-Traffic & Traffic \\ \hline \hline
         Non-Traffic & 83.0\% & 17.0\% \\ \hline
         Traffic & 9.0\% & 91.0\% \\ \hline
    \end{tabular}
    \caption{Confusion Matrix with 1000 traffic and 1000 non-traffic tweets.}
    \label{fig:confusionMatrix1}
\end{center}
\end{figure}	

Secondly, the classifier was trained with 1000 traffic and 9670 non-traffic tweets. Afterward it was tested with 500 traffic and 4835 non-traffic tweets. The metrics for this training are the following. 

Accuracy of the classifier:   0.862605

Traffic precision:\hspace{15.5 mm}            0.401019541206\\
Traffic recall:\hspace{21.2 mm}               0.944\\
Traffic F-measure:\hspace{12.8 mm}         0.562909958259\\

Non-Traffic precision:\hspace{7.2 mm}         0.993265993266\\
Non-Traffic recall:\hspace{13 mm}           0.854188210962\\
Non-Traffic F-measure:\hspace{4.6 mm}         0.918492160569\\

\begin{figure}[h!]
\begin{center}
    \begin{tabular}{| l || c | c | }
    \hline
          & Non-Traffic & Traffic \\ \hline \hline
        Non-Traffic & 85.4\% & 14.6\% \\ \hline
        Traffic & 5.4\% & 94.6\% \\ \hline
    \end{tabular}
    \caption{Confusion Matrix with 1000 traffic and 9670 non-traffic tweets.}
    \label{fig:confusionMatrix1}
\end{center}
\end{figure}
	
It can been observed from the confusion matrices, with the first training the classifier has been achieved a rather bad accuracy since 19\% of the non-traffic tweets are being classified wrong and 9\% of the traffic tweets are being classified wrong. On the other hand, with the second training the traffic tweets error has been almost halved to 5\% resulting a more accurate classification for the traffic tweets even if the global accuracy dropped by 1\%. That means the classifier may classified some traffic tweets as non-traffic but it classified much less non-traffic tweets as traffic.So ita has been decide from the team to train the Naive Bayes classifier with all the labelled data. Note that all the above results have been taken after the implementation of the normalization. During the first evaluation, before the pre-processing, the accuracy of the classifier was 77\%. However after the application of the normalization techniques the accuracy has been increased by 10\%!

A strong motivation for the creation of this project was the previous work of Dr.Luke Dicken in the same field. Having that in mind, is important to compare the accuracy of our classifier and the efficient of our server with his work. So in addition to the previous evaluation results, several metrics have been calculated in order to find the efficient of the classifier and the server. Those metrics are being presented in the figures below.

\begin{figure}[h!]
\begin{center}
    \begin{tabular}{| c | c | c | }
    \hline
        Total Number of Tweets  & Non-Traffic & Traffic \\ \hline 
        71036 & 68096(95.86\%) & 2940(4.14\%) \\ \hline
    \end{tabular}
    \caption{Comparison of Traffic and Non-Traffic Tweets.}
    \label{fig:metrics1}
\end{center}
\end{figure}
	
\begin{figure}[h!]
\begin{center}
    \begin{tabular}{| c | c | c | }
    \hline
        Geo-Tagged  & Geo-Tagged Genuine & Geo-Tagged from the 
Geolocation Resolver \\ \hline 
        314(10.68\%) & 136(4.63\%) & 178(6.05\%) \\ \hline
    \end{tabular}
    \caption{Tweets with geolocation from the 2940 traffic tweets.}
    \label{fig:metrics2}
\end{center}
\end{figure}


\begin{figure}[h!]
Overall inferred rates:
\begin{center}
    \begin{tabular}{| l || c | c | c |}
    \hline
        & Filtered & Traffic & Geo-Tagged and Traffic\\ \hline \hline
        each minute & 13.3 & 0.5 & 0.06 \\ \hline
		each 5 mins & 66.7 &  2.7 & 0.3 \\ \hline
		each hour & 800 & 32.7 & 3.5 \\ \hline
    \end{tabular}
    \caption{Averages are over total 5400 minutes (90 hours) of up-time.}
    \label{fig:metrics3}
\end{center}
\end{figure}

The corresponding metrics from Dr. Luke Dicken are being presented below.

Accuracy of the classifier:   0.818750

\begin{figure}[h!]
\begin{center}
    \begin{tabular}{| l || c | c | }
    \hline
          & Non-Traffic & Traffic \\ \hline \hline
        Non-Traffic & 85.8\% & 14.2\% \\ \hline
        Traffic & 22.3\% & 77.7\% \\ \hline
    \end{tabular}
    \caption{Confusion Matrix of Dr. Luke Dicken SVM classifier.}
    \label{fig:confusionMatrixLuke}
\end{center}
\end{figure}

\begin{figure}[h!]
Overall inferred rates:
\begin{center}
    \begin{tabular}{| l || c | c | c |}
    \hline
        & Filtered & Traffic & Geo-Tagged and Traffic\\ \hline \hline
        each minute & 6.3 & 3.6 & 0.11 \\ \hline
		each 5 mins & 32 &  18 & 0.5 \\ \hline
		each hour & 380 & 220 & 6.5 \\ \hline
    \end{tabular}
    \caption{Averages are over total 160000 minutes (111 days) of up-time.}
    \label{fig:metricsLuke}
\end{center}
It can be clearly seen that comparing to Dr. Luke's classifier; the Naive Bayes classifier sacrificed the quantity to achieve the quality. Our classifier labelled less tweets as traffic, but those tweets are rather accurate. That�s important because the application�s user prefer to have less tweets for each disruption, but these tweets have to be about traffic. 
\end{figure}

\subsubsection{Functional/Integration Testing}
To test boundaries between systems, functional testing is being applied. This enables us to test that the inputs and outputs 
of the system as a whole conform to the expected responses. The biggest of these boundaries is between the mobile application and the back
end service, where communication occurs over a REST API.

Many requests will be coming from the mobile application and it must be confirmed that the server interface can handle 
them but also return the expected results using the expected format. Initially the functional testing of this interface was performed by the command line tool `cURL'. 
But as development progressed the team was investigating moving to new tools to provide a better testing platform.

After researching, the Apache JMeter application was used \cite{ApacheJmeter}. This software is designed to load test 
functional behaviour and measure the performance of static and dynamic recourses. Using JMeter 
proved to be a very effective way to test the servers interface. Various test plans where created 
testing every REST endpoint request available to use through the API. These plans where created for 
both the server and the mock server that are running on ports 55004 and 55003 accordingly. More 
tests were implemented to ensure that the server was returning the correct messages and response 
codes when it was encountering an error. In addition, invalid requests to the server had also been 
checked for error handling and if the error messages were displayed correctly into the screen. For 
each request, the content-type was also checked if it evaluates as mimetype of `application/json'. This process was really 
helpful because it was very easy later on to check whether new features implemented or code 
refactoring were actually breaking the API. All the test plan configuration settings have been saved 
in the project repository. 

\subsubsection{Unit Testing} 
Unit testing in the project became an issue as the project developed. This was
due to a number of reasons which largely stemmed from the project's research
oriented nature.

Much of the data analysis code on the server side was written experimentally
initially and as it developed was included in the production server. With this
type of code it is very difficult to define the requirements in terms of test
driven design before a method or class is initially developed.

Time during this project was a major issue, near the end of the project a team
decision was taken to concentrate on implementing the requirements over
spending time writing unit tests.
