\subsection{Development approach} 
To begin identifying an appropriate development methodology, it was necessary to firstly understand 
the project requirements. The project requirements and features were agreed over a number of 
meetings with the primary stakeholders.\\ 
The two most preferred development methodologies were 
Scrum and XP(Extreme Programming). Scrum is an agile project management technique that focuses more 
on the management of software development projects. The product is completed in a series of one to 
four week iterations, or sprints as they are called. Before each sprint, a planning meeting is held 
to determine which features will be implemented during that sprint. Similarly, XP is an agile 
methodology which is designed for small, co-located teams aiming to get quality and productivity as 
high as possible. It does this through the use of rich, short, informal communication paths with 
emphasis on skill, discipline and understanding at the personal level, minimizing all intermediate 
work products.\\ 
It was decided amongst the group that combining characteristics form both methods 
would be the most beneficial way, merely because Scrum focuses on the project management whereas XP 
on the programming part.\\ 
Managing efficiently the development lifecycle of the project was a very 
important task for the team to work properly. Meetings with the supervisor took place each week 
where the project progress was thoroughly discussed as well as more feasible features to be 
implemented. In addition, various issues were brought up in order to provide solutions. Except from 
the meetings with the supervisor, the team had its own weekly meeting for keeping up with what 
everyone has been doing for the last week. New tasks were also assigned to members of the team. For 
every meeting an agenda was stored in an online document containing information about the team 
members absent, location, action items and topics to be discussed. In the end of the meetings tasks 
were assigned, managed or archived using an online visual board, similar to Scrum Board, to 
encourage a more agile development.\\ 
As regards to the XP approach, it was proved that adopting 
some of its techniques would significantly improve the development process. More specifically, pair 
programming was effectively put in use. Team members where working in pairs whenever possible to 
develop a single feature. Additionally, XP promotes test-driven development. As mentioned in the 
first report, testing would be rather difficult due to the nature of the project being partly 
research based. However, essential tests were implemented in the server to validate the classifier 
as well as blackbox tests for the correctness and responsiveness of the rest API server. \\ 
Throughout the development, various technologies had to be used and implemented. According to each 
members skills these elements were divided in such ways to effectively use the members previous 
experience. The team was also split to different design aspects of the project; however, more on the 
team structure will be discussed further on.\\ 
Lastly, for achieving parallel development between 
the mobile client application and the server, a mock server API was created, as mentioned in earlier 
reports that was accepting http requests and was returning static data sets to the client. 

\subsection{Testing} 
\subsubsection{Classifier Evaluation} 
\subsubsection{Functional/Integration Testing}
One of the most important parts of the project was the server API interface. A lot of requests will 
be coming from the mobile application and it must be confirmed that the server interface can handle 
them but also return the expected results using the expected format. In order to test the server 
interface the Apache JMeter desktop application was used. This software is designed to load test 
functional behaviour and measure the performance of static and dynamic recourses. Using JMeter 
proved to be a very effective way to test the servers interface. Various test plans where created 
testing every GET or POST request available to use through the API. These plans where created for 
both the server and the mock server that are running on ports 55004 and 55003 accordingly. More 
tests were implemented to ensure that the server was returning the correct messages and response 
codes when it was encountering an error. In addition, invalid requests to the server had also been 
checked for error handling and if the error messages were displayed correctly into the screen. For 
each request, the content-type was also checked if it evaluates as JSON. This process was really 
helpful because it was very easy later on to check whether new features implemented or code 
refactoring were actually breaking the API. All the test plan configuration settings have been saved 
in the project repository as a .jmx file. 

\subsubsection{Unit Testing} 

