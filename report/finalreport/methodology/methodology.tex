\subsection{Development approach} 
To begin identifying an appropriate development methodology, it was necessary to firstly understand 
the project requirements, timescale of the project and the team members experience. The project requirements and features were agreed over a number of 
meetings with the primary stakeholders.

The two most preferred development methodologies were 
Scrum and XP(Extreme Programming) which are both Agile methodologies. Scrum is an agile project management technique that focuses more 
on the management of software development projects. The product is completed in a series of one to 
four week iterations, or sprints as they are called. Before each sprint, a planning meeting is held 
to determine which features will be implemented during that sprint. Similarly, XP is an agile 
methodology which is designed for small, co-located teams aiming to get quality and productivity as 
high as possible. It does this through the use of rich, short, informal communication paths with 
emphasis on skill, discipline and understanding at the personal level, minimizing all intermediate 
work products. 

It was decided amongst the group that combining characteristics from both methods 
would be most beneficial, merely because they are complementary. Scrum focuses on the project management 
whereas XP focuses on programming.\cite{ScrumXP}

Managing efficiently the development lifecycle of the project was a very 
important task for the team to work effectively. Meetings with the supervisor took place each week, 
the project progress was discussed as well as other possible features that could be implemented. 
In addition, various issues were constantly being brought up in order to provide solutions. Apart from 
the meetings with the supervisor, the team had its own weekly meeting for keeping up with 
everyone's progress for the previous week. New tasks were also assigned to members of the team. For 
every meeting an agenda was stored in an online document containing information about the team 
members absent, location, action items and topics to be discussed. In the end of the meeting tasks 
were assigned, managed or archived using an online visual board, similar to the Scrum Board, to 
encourage a more agile development.

As regards to the XP approach, it was proved that adopting 
some of its techniques would significantly improve the development process. More specifically, pair 
programming was effectively put in use. Team members were working in pairs whenever possible to 
develop a single feature that seemed to be difficult. As a result, any time the two find a section of code 
that appears hard to understand or overly complex, they are to revise it, constantly simplifying and improving it. 
Additionally, XP promotes test-driven development. As mentioned in the 
first report, testing would be rather difficult due to the nature of the project being partly 
research based. However, essential tests were implemented in the server to validate the classifier 
as well as blackbox tests for the correctness and responsiveness of the rest API server. 

Throughout the development, various technologies had to be used and implemented. According to each 
members skills these elements were divided in such a way to effectively use the members previous 
experience. The team was split to concentrate on different design aspects of the project; however, more on the 
team structure will be discussed further on.

Lastly, for achieving parallel development between the mobile client application and the server, 
a mock server API was created, as mentioned in earlier reports that was accepting http requests and 
was returning static data sets to the client. 

\subsection{Testing} 

\subsubsection{Functional/Integration Testing}
To test boundaries between systems, functional testing is being applied. This enables us to test that the inputs and outputs 
of the system as a whole conform to the expected responses. The biggest of these boundaries is between the mobile application and the back
end service, where communication occurs over a REST API.

Many requests will be coming from the mobile application and it must be confirmed that the server interface can handle 
them but also return the expected results using the expected format. Initially the functional testing of this interface was performed by the command line tool `cURL'. 
But as development progressed the team was investigating moving to new tools to provide a better testing platform.

After researching, the Apache JMeter application was used \cite{ApacheJmeter}. This software is designed to load test 
functional behaviour and measure the performance of static and dynamic recourses. Using JMeter 
proved to be a very effective way to test the servers interface. Various test plans where created 
testing every REST endpoint request available to use through the API. These plans where created for 
both the server and the mock server that are running on ports 55004 and 55003 accordingly. More 
tests were implemented to ensure that the server was returning the correct messages and response 
codes when it was encountering an error. In addition, invalid requests to the server had also been 
checked for error handling and if the error messages were displayed correctly into the screen. For 
each request, the content-type was also checked if it evaluates as mimetype of `application/json'. This process was really 
helpful because it was very easy later on to check whether new features implemented or code 
refactoring were actually breaking the API. All the test plan configuration settings have been saved 
in the project repository. 

\subsubsection{Unit Testing} 
Unit testing in the project became an issue as the project developed. This was
due to a number of reasons which largely stemmed from the project's research
oriented nature.

Much of the data analysis code on the server side was written experimentally
initially and as it developed was included in the production server. With this
type of code it is very difficult to define the requirements in terms of test
driven design before a method or class is initially developed.

Time during this project was a major issue, near the end of the project a team
decision was taken to concentrate on implementing the requirements over
spending time writing unit tests.
