\subsection{Team Structure}
Project management is a necessary part of group  project. Trello is a easy tool for us to plan this project well. We set four statements for anything we want to do: 'To Do', 'Started', 'Review' and 'Done'. At the same time, six labels are built with different colours to identify their kinds of tasks: 'Data', 'Documentation', 'Server', 'Android App', 'Text Analysis' and 'Meeting Prep'. All tasks will be discussed at week meeting and assigned to 'To Do' list. When team member who is responsible for the task will drag this task card to 'Started' list to inform others that this task is being done. After this task has been finished, team member will move this task to 'Review' list to tell others to check and evaluate it. In the next meeting, team members will evaluate all tasks in the 'Review' list and decide to move them to 'Done' list. Each card is an action item of choosing which is as simple as an item on a shopping list.\\
Github is designed very well for group developing. A repository called 'Twitter for Traffic' is built in Github.We separate project into several parts. We create a 'classifier' file for data mining and 'data\_acquisition' file to retrieve tweets and TfL data. 'mobile\_client' folder is built for Android application development. 'mock\_server' and 'server' folders are platform providing data for mobile client. A folder called 'scrapbook' is created for each members to store source codes and test each unit function of this project. Each team member track file in project and periodically commit the state of the project when a saved point wanted. Then we can share that history with other developers for collaboration, merge between everyone's work, and compare or revert to previous versions of the project or individual files. What's more, wiki tab can enable team members to add any information that which is useful to project but others don't know.
\subsection{Collaboration Tools}
At the early stages of the project, all areas were investigated to determine their feasibility and to estimate the amount of work which should be done. Each team member did research on a different part of the specification in order to establish how many steps need to be taken to get the goal and how long. The group then discussed which team member would be responsible for different tasks. Every team member is expected to contribute at least 10 hours per week on this group project.The initial arrangement was listed below, shown as a table.\\
\begin{tabular}{|c|p{11.5cm}|}
\hline
Name&Task\\
\hline
Porfyrios Vasileiou&Create mock server to provide JSON data for communication with mobile application. Process mobile client report. Test server using black box Jmeter. Store optimized route for mobile user.\\
\hline
Marianna Polatoglou&Identify tweet clusters, investigate and identify classification, create second Tweet Classifier(svm).\\
\hline
Afxentios Hadjiminas&Store and categorise social data from Twitter and TfL. Identify traffic disruptions from Twitter, and create Naive Bayes classifier.\\
\hline
Panagiotis Tsirigotis&Store disruptions form TfL and have a current representation of the event. Configure database scripts. Create Twitter user blacklist.\\
\hline
John Flanagan&Mobile application design and development: report events, present a list of classified and grouped tweets, show local disruption map, show stored routes and GUI design.\\
\hline
Hanguang Zhou&Geocoding of messages without explicit location. Implement Soundex in geocoding.\\
\hline
\end{tabular}