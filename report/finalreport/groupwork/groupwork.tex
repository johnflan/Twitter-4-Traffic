\subsection{Team Structure}
At the early stages of the project, all areas were investigated to determine their feasibility and to estimate the amount of work which should be done. Each team member did research on a different part of the specification in order to establish how many steps need to be taken to get the goal and how long. The group then discussed which team member would be responsible for different tasks. Every team member is expected to contribute at least 10 hours per week on this group project.\\
\subsection{Collaboration Tools}
Project management is a necessary part in group project. Trello is an easy tool for us to plan this project well. We set four statements for anything we want to do: 'To Do', 'Started', 'Review' and 'Done'. At the same time, six labels are built with different colours to identify their kinds of tasks: 'Data', 'Documentation', 'Server', 'Android App', 'Text Analysis' and 'Meeting Prep'. All tasks will be discussed at week meeting and assigned to 'To Do' list. When team member who is responsible for the task will drag this task card to 'Started' list to inform others that this task is being processed. After this task has been finished, team member will move this task to 'Review' list to tell others to check and evaluate it. In the next meeting, team members will evaluate all tasks in the 'Review' list and decide to move them to 'Done' list. Each card is an action item of choosing which is as simple as an item on a shopping list.\\
Github is used as our revision control software to manage source-code and documentation. It enables multiple team members to work on a set of files and ensures that changes will not get lost. A repository called 'Twitter for Traffic' is built in Github. Project is divided into several parts. We create a 'classifier' file for data mining and 'data\_acquisition' file to retrieve tweets and TfL data. 'mobile\_client' folder is built for Android application development. 'mock\_server' and 'server' folders are platform which is providing and processing data for mobile client. A folder called 'scrapbook' is created for each members to store source codes and test each unit function. Each team member tracks files in project and periodically commit the state of the project when a saved point wanted. Then we can share that history with other developers for collaboration, merge between everyone's work, and compare or revert to previous versions of the project or individual files. What's more, 'WIKI' tab can enable team members to add explanation for specific feature as same as Wikipedia. From the beginning of the project it was agreed that all data will exist outside of the repository. This was enforced by policy and by using a '.gitignore' file of the repository to list file types to be not tracked.
