\subsection{Team Structure} 
Before starting working on the project, the team structure had to be defined. Most importantly, we had to 
agree who would be the leader of the group whose main responsibility was coordinating the tasks amongst the 
team members. At the early stages of the project, all features were researched in order to 
determine their feasibility and to estimate the amount of work required for each one. Next, according to 
each members previous experience and expertise, the team was divided into 3 subgroups, each one assigned to 
a different aspect of the project. Only one person was responsible for designing the client application, while
we provisioned the team manpower to the server. Two members were allocated to creating and training 
the classifier and two more focused on the implementation of the server. The last person was rotating through 
all the aspects of the project, providing help when needed. Every team member was expected to contribute 
at least 10 hours per week on this group project.

\subsection{Collaboration Tools} 
Project management is a vital part of any project. There exist a series of tools that enable effective team 
collaboration during a project. One of these tools is \emph{Trello} which gives access to a visual board 
that displays the ongoing tasks. This concept is very similar to the Scrum board. All these tasks 
are represented as cards with labels, priorities and deadlines assigned to 
team members. In this visual board we have the lists for separating the tasks such as `To Do', `Started', `Review' and `Done this week'. 
Additionally, six labels were created with different colours to identify their kinds of tasks like data, 
documentation, server, android app, text analysis and meeting preparation. Future tasks were 
discussed at the weekly group meeting and assigned to the `To Do' list. A team member who started a task that he was responsible of, 
would drag this card to the `Started' list to inform the rest group members that this task was currently being processed. 
After a task was finished, the card was moved to the `Review' list so other members would evaluate it. 
On every group meeting, reviewed tasks were transferred to the `Done this week' list and in the end of the meeting they were
archived.

Another tool that has been used for this project was Github. Github is used as our revision control system to manage 
the source-code and documentation. It enables multiple team members to work on a set of files, 
ensuring that changes will not get lost. Other features include version tracking, verifiable histories, integration with deployment, 
test and code review systems and the ability to add rules to enforce corporate policy and process.

When a team member makes changes to a file in the project, his next step is to commit it and push it to the repository. 
This updates the state of the project and the other team members can pull and receive the updated project repository. 
Github stores the history of all the changes so that individuals can compare or revert 
to previous versions of the project. Furthermore, Github provides a `Wiki' feature that enables team members 
to add description or instructions for specific components that are implemented in the project.

From the beginning of the project it was agreed that all data will exist outside of the repository. This was enforced by 
policy and by using a `.gitignore' file of the repository to list file types to be not tracked.


% A repository called 'Twitter for Traffic' is created in 
% Github. The project is divided into several parts. We create a 'classifier' file for data mining and 
% 'data\_acquisition' file to retrieve tweets and TfL data. 'mobile\_client' folder is built for 
% Android application development. 'mock\_server' and 'server' folders are platform which is providing 
% and processing data for mobile client. A folder called 'scrapbook' is created for each members to 
%store source codes and test each unit function.
