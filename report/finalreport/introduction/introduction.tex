Internet connected mobile phones and micro-blogging platforms like Twitter have
been providing researchers with a vast and interesting source of real-world
data. This data has been used to analyse trends, identify concepts and to
augment traditional curated data providing a level of social knowledge to the
topic. 

Twitter for traffic aims to take these concepts to provide the mobile user with
an application for assessing traffic disruptions as they evolve for London
city. To provide this insight, the application will present the user with
curated traffic disruptions augmented with social knowledge of the event
harvested from a social network. 

In addition to this curated list of traffic disruptions, the concept of
identifying emerging disruptions from this social data will be explored, with
the aim of identifying clusters of disruption reports to discover new
disruption events.

To promote the of reporting the traffic conditions on a social network, it will
also be necessary to provide a fast, simple and descriptive mechanism for the
user to manually contribute their observations.

Aspects of this project include data mining, social network analysis, mobile
application development, document classification and geographical information
systems. Analysing the social and curated data feeds to identify new disruption
events will be a particularly interesting undertaking. From the onset there are
many unknowns relating to the data analysis, including the quantity of traffic
related messages and quality of the content.
