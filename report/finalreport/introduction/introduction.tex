City planners are struggling to acquire actionable real-time data on traffic
conditions. This data is especially important for incidents which may translate
into traffic disruptions on already over capacity road networks. Cities have
experimented with traffic cameras, dynamic traffic light control and more
traditional traffic based radio stations. Initiatives such as the `Smarter
Cities’ concept from IBM are actively researching this problem
\cite{website:smarter_cities}.

Internet connected mobile phones and micro-blogging platforms like Twitter have
been providing researchers with a vast and interesting source of real-world
data. This data has been used to analyse trends, identify concepts and to
augment traditional curated data providing a level of social knowledge to the
topic.

Twitter for traffic aims to take these concepts to provide the mobile user with
an application for assessing traffic disruptions as they evolve and provide
city planners with this real-time knowledge.

To provide this insight, the application will present the user with curated
traffic disruptions augmented with social knowledge of the event harvested from
a social network.  In addition to this curated list of traffic disruptions,
identifying emerging disruptions from this social data will be explored, with
the aim of identifying clusters of disruption reports to discover new
disruption events.

To promote the of reporting the traffic conditions on a social network, it will
also be necessary to provide a fast, simple and descriptive mechanism for the
user to manually contribute their observations.

There are a number of interesting challenges that comprise this task. Initially
we must identify tweets that discuss road traffic. For these identified traffic
tweets to be of value we must be able to infer some geographic location for
each tweet. It seems that a small percentage of tweets contain an explicit
geographic location, so other methods for finding this location must be
explored. To identify events from tweets alone, we must investigate geographic
clusters of tweets which discuss traffic. There are a number of problems in
detecting valid traffic effecting events, these include sufficient data,
timeliness of the tweets and  ‘normal’ levels of background chatter.

From the onset there are many unknowns relating to the data analysis aspect of
this project, including the quantity of traffic related messages and quality of
the content. Aspects of this project include data mining, social network
analysis, mobile application development, document classification and
geographical information systems.
