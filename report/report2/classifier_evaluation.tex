Evaluation of tweets classification

In order to retrieve information about traffic disruptions from social data, there is a need to classify the incoming information.
Still, there are a lot of classification techniques, as well as different parameters in the same techniques, and there is a choice to be made. These classification methods use a train set, which is a set of text that has been labelled by the team, for their training process. Also they use a labelled test set, which has to be distinct from the train set, to evaluate this training to data they have not been directly trained to recognise. Note that there is a chance the classifier will become more accurate in the train set and less accurate in the test set with some parameter changes. This is when we have reached an "overfitting" to the train set.
To test the efficiency of the classifiers there is a built-in function of the package NLTK nltk.classify.accuracy(), which has been used.

However more techniques will be used, when the dataset will be further increased. The first of these methods is the K-Fold Cross Validation. The dataset is split each time into K equally sized subsets, training and testing datasets, and then in n-th iteration (n=1..k) the n-th subset(testing set) is used for testing the classifier that has been built on all other remaining subsets. To generate multiple samples from a single sample, an alternative to cross-validation is the Bootstrap that generates new samples by drawing instances from the original sample with replacement. Another method is the Confusion Matrix which is a visualization tool typically used to present the results attained by a learner. Each column of the matrix represents the instances in a predicted class, while each row represents the instances in an actual class.For this purpose, the NLTK package provides the function nltk.ConfusionMatrix (). Finally, the Precision and Recall Rates can be calculated in order to ensure the results from the previous method. The recall and the precision can be derived easily from the confusion matrix.