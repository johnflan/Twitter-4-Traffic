Evaluation of tweets classification

Machine learning algorithms induce classifiers that depend on the training set, and there is a need 
for evaluation and statistical testing to assess the expected error rate of a classification algorithm,and even compare the expected error rates of two classification algorithms to be able to say which one is better. Evaluation can also be used as a guide for future improvements on the model. In order to evaluate the classifier several techniques have been used. The first technique is to generate a test set of tweets which their labels are already known. This test set has to be distinct from the train set which has been used to train the classifier. Afterward this test set is being classified by the classifier and the labels, that it decides, are being compared with their correct labels. The second technique is to calculate the accuracy of the classifier which measures the percentage of inputs in the test set that the classifier correctly labelled. To accomplish this the build-in function of the package NLTK nltk.classify.accuracy() has been used.
 
However more techniques will be used, when the dataset will be further increased, in order to get more accurate evaluations and avoid possible �overfitting�. The first of these methods is the K-Fold Cross Validation. The dataset is split each time into K equally sized subsets, training and testing datasets, and then in n-th iteration (n=1..k) the n-th subset(testing set) is used for testing the classifier that has been built on all other remaining subsets. To generate multiple samples from a single sample, an alternative to cross-validation is the Bootstrap that generates new samples by drawing instances from the original sample with replacement. Another method is the Confusion Matrix which is a visualization tool typically used to present the results attained by a learner. Each column of the matrix represents the instances in a predicted class, while each row represents the instances in an actual class.For this purpose, the NLTK package provides the function nltk.ConfusionMatrix (). Finally, the Precision and Recall Rates can be calculated in order to ensure the results from the previous method. The recall and the precision can be derived easily from the confusion matrix.

