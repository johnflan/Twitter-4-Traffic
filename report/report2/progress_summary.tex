The project is currently in Week 5 of the 8 Week schedule, to date the team has
remained on course with the project schedule. A number of team decisions have
helped to maintain this progress. The most important of which are the use of a
scrum board for keeping track of tasks in progress, the division of work to
play to members strengths and the separation of server and mobile development
tasks.

Although the team is currently on track, it is from this point onwards that
meeting the deadlines will become increasingly difficult. Up to this point we
were implementing the ‘low hanging fruit’ functionally, the outstanding tasks
like “Geocoding of messages without explicit locations”, “Enhance
classification and clustering algorithms” and “Disruptions for Stored Routes”
have some inherent difficulties.

\subsection{Server development}

The server has been configured with a geographical database and we are
currently collecting Transport for London and Twitter data for analysis.

A small unforeseen difficulty with the storage of Transport for London data was
that all location data was encoded in with UK Grid references and
latitude/longitude references were required for our application. Fortunately
Ordnance Survey provide a technique for translating between the coordinate
systems which was successfully implemented.

We have manually labelled a subsection of these tweets to train a
classifier to identify tweets about vehicular traffic. Labelling of the tweets
has proven to be a time consuming process. From Twitter we request tweets that
match any of about 15 terms, even with this restriction of the domain
we are seeing only about 1\% of tweets with relevant traffic content.

It is planned in the next week to bootstrap the initial classifier with the manually
labeled set. Bootstrapping enables us to train an initial classifier from a
small dataset. The classifier can then be used to discover so the group can manually label
positive results in order to create a larger training set. With this larger
training set it is expected we can then create a more accurate classifier.
There is an inherent danger in this technique, it is possible to `overfit' the
classifier to our initial training data, and then amplify the initial `overfit'. This
is something we will have to actively monitor.

A basic Naive Bayes classifier is currently in development and is being trained
using these manually labeled tweets. During the classification  of the tweets
some non-traffic tweets were being labelled as traffic. After some 
research the team has identified that a lot of the features-words were having
very low information gain(informativeness). To overcome this obstacle the
stop-words have been found and have been filtered from the feature set.The stopwrods 
are a set of words that doesn't provide any valuable information to the classifier.  In
addition, because of the nature of the tweets and in order to increase the
performance of the classifier significant unordered bigrams have been included. 

Finally, a module has also been written to identify timely geographical
clusters of traffic tweets in an attempt to predict new traffic disruption
events. 


\subsection{Mobile application development}

In order to minimise dependency on the server development for the mobile
application, a  mock web-server was created to serve sample data in the agreed
format. The mobile application development began with the creation of a number
lo-fidelity UI prototypes, once the functionality of the application was agreed
a hi-fidelity design was created and implemented. Currently the application is
able to request traffic events for the current location and display them in a
list, on clicking on a traffic event a list of tweets for the event are
requested and displayed to the user. 

\subsection{Specification updates}

Features and requirements suggested in the initial report remain unchanged,
with the same priority and due dates. However, a number of additional features
have been proposed for the mobile application during supervisor meetings. One
of them is allowing the user to toggle tweets visibility on the map according
to their geographical location. Another feature is enabling the user to rate
all the tweets and as a result the higher ranked ones will be shown on the map.

As for the server application an extra feature will be to store the locations
of traffic cameras. When a user selects a traffic event on the map he will be
presented with links to nearby cameras.


\begin{center}
\begin{tabular}{ | p{9cm} | c | c | p{1.8cm} | }
\hline
\multicolumn{4}{|c|}{\textbf{Mobile Application}} \\ \hline
\textbf{Feature} & \textbf{Priority} & \textbf{Feasibility} & \textbf{Due date}
\\ \hline
\textbf{Present tweets on the map}\newline
Display high ranked tweets on the map. & & & Week 6 \\ \hline
\textbf{Present traffic camera for events}\newline
Present traffic camera for events
If a traffic event occurs in the vicinity of a traffic camera, present the user
with a camera icon. If they click on the icon they are presented with a live
view of the event. &  &  & Week 7 \\ \hline
\end{tabular}
\end{center}

\begin{center}
\begin{tabular}{ | p{9cm} | c | c | p{1.8cm} | }
\hline
\multicolumn{4}{|c|}{\textbf{Server Application}} \\ \hline
\textbf{Feature} & \textbf{Priority} & \textbf{Feasibility} & \textbf{Due date}
\\ \hline
\textbf{Store traffic cameras by geolocation}\newline
Store a list of public traffic cameras in the database by their geographic
location. When returning traffic disruptions in the vicinity of a camera,
attach a link to the most relevant camera. &  &  & Week 7 \\ \hline
\end{tabular}
\end{center}
