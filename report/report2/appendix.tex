\begin{center}
\begin{tabular}{ | p{9cm} | c | c | p{1.8cm} | }
\hline
\multicolumn{4}{|c|}{\textbf{Mobile Application}} \\ \hline
\textbf{Feature} & \textbf{Priority} & \textbf{Feasibility} & \textbf{Due date} \\ \hline
\textbf{Report events (Done)}\newline
Report traffic disruptions through the mobile application on Twitter. & 1 & 2 & Week 3 \\ \hline
\textbf{Present a list of classified and grouped tweets}\newline
Present to the user tweets identified as describing a traffic disruption,
clustered around an event. & 2 & 4 & Week 6 \\ \hline
\textbf{Show local disruption map}\newline
Show positions of known disruptions on a map. Enable the user to click on an event for further information. & 3 & 5 & Week 6 \\ \hline
\textbf{Stored Routes}\newline
Find and show disruptions for user defined stored routes. & 4 & 7 & Week 7 \\ \hline
\textbf{Present tweets on the map (New)}\newline
Display high ranked tweets on the map.& 6 & 7 & Week 6 \\ \hline
\textbf{Present traffic camera for events (New)}\newline
If a traffic event occurs in the vicinity of a traffic camera, present the user
with a camera icon. If they click on the icon they are presented with a live
view of the event. &  8 & 4 & Week 7 \\ \hline
\multicolumn{4}{|c|}{Minimum specifications for the mobile application are
features including and below priority 3.} \\ \hline
\end{tabular}
\end{center}

\begin{center}
\begin{tabular}{ | p{9cm} | c | c | p{1.8cm} | }
\hline
\multicolumn{4}{|c|}{\textbf{Server Application}} \\ \hline
\textbf{Feature} & \textbf{Priority} & \textbf{Feasibility} & \textbf{Due date}
\\ \hline
\textbf{Store disruptions from TfL (Done)} \newline
Create a geographic database of live events from TfL. Evolve these database events with data from the feed, in order to have a current representation of the event. & 1 & 3 & Week 2 \\ \hline
\textbf{Store and categorise social data (Done)} \newline
For tweets with an explicit location, store them in a geographical database
table. And those tweets without this explicit information store them separately for later
analysis. & 2 & 2 & Week 2 \\ \hline
\textbf{Process mobile client reports} \newline
Take twitter messages reported from the mobile client and process these in a
separate pipeline. & 3 & 2 & Week 6 \\ \hline
\textbf{Classify tweets from Twitter with a simple classifier (Done)} \newline
Train a document classifier using a manually labeled training set to 
identify traffic related tweets. & 4 & 5 & Week 5 \\ \hline
\textbf{Identify traffic disruptions from Twitter} \newline
Inspect incoming traffic tweets and determine new non-TfL disruptions from
those. & 5 & 5 & Week 5 \\ \hline
\textbf{Geocoding of messages without explicit locations} \newline
Resolve geographic locations extracted from the message context. & 6 & 7 & Week 6 \\ \hline
\textbf{Enhance classification and clustering algorithms} \newline
From insight and data gained from initial classification and clustering
techniques, attempt to improve the accuracy of the results. & 7 & 9 & Week 7 \\ \hline
\textbf{Store traffic cameras by geolocation(New)}\newline
Store a list of public traffic cameras in the database by their geographic
location. When returning traffic disruptions in the vicinity of a camera,
attach a link to the most relevant camera. &  8 &  4 & Week 7 \\ \hline
\multicolumn{4}{|c|}{Minimum specifications for the server application are
features including and below priority 4.} \\ \hline
\end{tabular}
\end{center}