Management of source-code, documentation and binary files can be tedious for an individual, but when you consider a team working on a set of files, tracking changes and version becomes close to impossible.

While many different approaches are used to track files, most revision control systems offer
similar basic functionality. This main functionality could be described under
the title of 'collaboration', enabling multiple individuals to work on a set of
files while ensuring changes don't get lost and enabling tracking of versions.
Modern systems offer powerful features such as verifiable histories, ability to integrate with deployment, test and code review systems and the ability to add rules to enforce corporate policy and process.

The chosen revision control system needs to be free and preferably open-source,
work well in a disconnected environment, have good documentation and also
beneficial if this tool is in common use. Git\cite{website:git_scm}, the
open-source distributed revision control system used by many major projects
including the Linux Kernel, Google's Android and Eclipse seemed to meet the
teams requirements. Additionally, members of the team have previous experience using the tool, which will lessen the initial learning curve.

The distributed nature of Git means a centralised 'server' is not necessary,
but for convenience Github\cite{website:github} an online hosting service for
Git is to be utilised. In addition to hosting the repository, Github offers a
simple user interface for browsing the history, bug-tracking, wiki, and
is free for public projects with less than 300mb of history. However, using
Github means we lose the ability to interface with 'git hooks' on the server. 

From the beginning of the project it was agreed that all data will exist
outside of the repository. This was enforced by policy and by using a
'.gitignore' file in the root of the repository to list file types to be not tracked.
